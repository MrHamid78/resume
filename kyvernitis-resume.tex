\documentclass[]{kyvernitis-resume}
\usepackage{enumitem}

\fullname{Hamidreza}
% \jobtitle{Sofware Engineer}

\begin{document}
\resumeheader
{\linkedin{hamidreza-kouchakzadeh}}
{\email{hamidrezakouchakzadeh@gmail.com}}
{\github{MrHamid78}}
{\phone{+98 9330243474}}
% {\website{example.com}}
{\nationality{Iranian}}

\begin{section}{About me}
Experienced PHP developer with a deep understanding of the language and its ecosystem. Proficient in building robust web applications and implementing complex features using PHP and frameworks like Symfony. Skilled in database management with MySQL and proficient in front-end technologies such as HTML, CSS, JavaScript, and jQuery. Detail-oriented and dedicated to writing clean, efficient, and maintainable code. Continuously learning and staying updated with the latest PHP trends and best practices. Committed to delivering high-quality solutions and providing excellent user experiences.
\end{section}


\begin{section}{Work Experience}
    \begin{subsection}{Pasargad Electronic Payment Co.}{Software Developer}{June 2022 -- PRESENT}{Tehran, Iran}
     		\item Subsidiary of Pasargad Bank, specializing in electronic payment services.
		\item Gained experience working with large-scale software systems within the company.
		\item Transitioned from PHP to Java and Spring Boot framework during this period.
		\item Also worked with Angular framework in the front-end development as per project requirements.
		\item Familiarized myself with Agile methodologies and worked with Elasticsearch and RabbitMQ for request queue management.
		\item Contributed to several projects within the company, including:
		
		\begin{itemize}[label={}, leftmargin=1cm]
		  \item Internet Payment Gateway (IPG): Developed and maintained an online payment gateway system.
		  \item Traffic Violations System: Worked on the development and enhancement of a system for managing traffic violations.
		  \item Feeda (Foreigners' Unique Code): Participated in the development of a system for generating unique codes for foreigners.
		  \item Seta (Special Invoice Generation System): Collaborated on the creation of a specialized invoice generation system.
		\end{itemize}

    \end{subsection}

 \begin{subsection}{RIECO.}{Software Developer}{Dec 2020-- Nov 2021}{Alborz, Iran}
       	\item Specialized in the development and production of software solutions for the oil and gas industry.
	\item Developed maintenance software, Integrated Decision-Making Systems (IDMS), and Engineering Document Control Software (EDMS).
	\item Collaborated with major engineering companies such as Rampco, Persian gulf petrochemical industries , and Apadana Petrochemical, providing them with our software solutions.
	\item Focused on studying software design patterns, particularly MVC, to meet the complexity of projects.
	\item Conducted back-end development using PHP, Symfony libraries, and MySQL database ; also Implemented Redis caching for improved performance.
	\item Contributed to a parallel project using Golang, ensuring its integration with our main software.
	\item Coordinated the work of the programming team with other departments.

    \end{subsection}
    
    \begin{subsection}{Koleh Poshti}{Project Manager \& Back-End Developer}{Dec 2019 -- July 2020}{Alborz, Iran}
        	\item Worked in the tourism industry for Koleh Poshti, a company specializing in organizing outbound tours.

	\item Managed and implemented a software project for customized tours in Turkey, where customers could personalize their travel experiences from accommodation to specific destinations and activities during their stay.

	\item The project received investment and support from 100 Startups, a capital investment event, and later secured funding from NovinTech.

	\item Unfortunately, due to the COVID-19 pandemic and the economic challenges that impacted society, the startup was forced to halt operations.

	\item Utilized PHP and MySQL database for the project development in Koleh Poshti.

	\item Familiarized myself with the WordPress platform to facilitate the migration of the company's previous website to the new system.

	\item Engaged in business aspects of the company, collaborating with Zavie to attract investors.
    \end{subsection}
    
    \begin{subsection}{Javanpardaz Alborz}{Project-based Work}{Oct 2016 - June 2017}{Alborz, Iran}
	\item Engaged in a project at Javanpardaz Alborz, primarily focusing on the development of the Sama software (abbreviated as "Sandogh-e Mekanize Amlak" in Persian), which was a real estate management system.
	\item Acquired proficiency in C\# programming language and utilized Telerik components to design the relevant Windows forms.
	\item The workflow of the software involved calculating commissions for real estate consultants in each transaction and generating invoices for the parties involved. Additionally, it featured expense calculation, income tracking, and the ability to generate reports for tax purposes to be submitted to the Tax Administration.
	\item Other notable features of the software included hardware lock protection and file management for real estate consultants.
	\item Each real estate consultant had a personalized user area on the web page, requiring activation of the software through a serial number and password, which had to be registered in their user area.
	\item The web panel was developed using PHP for the back-end and JavaScript for the front-end.
	\item The Sama software remained active for several years in provinces such as Khuzestan, Alborz, and certain areas of Tehran.
    \end{subsection}


 \begin{subsection}{Amoozesheyar (Educational Website)}{Teaching}{Mar 2014 - Aug 2014}{Remote , Iran}
	\item Engaged in teaching and instructing web design, programming, and implementing WordPress themes on the Amoozesheyar website.
	\item Provided tutorials on setting up a blogging system using WordPress, which was a unique offering in Iran during that time.
	\item Published instructional materials on the website during this period.
	\item Part 1:  \href{https://www.daneshjooyar.com/%d9%82%d8%b3%d9%85%d8%aa-%d9%86%d9%87%d8%a7%db%8c%db%8c-%d8%b3%d8%b1%db%8c-%d8%a2%d9%85%d9%88%d8%b2%d8%b4%db%8c-%d8%b7%d8%b1%d8%a7%d8%ad%db%8c-%d9%82%d8%a7%d9%84%d8%a8-%d9%88%d8%b1%d8%af%d9%be%d8%b1/}{Link to the published tutorials on WordPress theme design} 
	\item Part 2: \href{https://www.daneshjooyar.com/%d8%a2%d9%85%d9%88%d8%b2%d8%b4-%d8%b7%d8%b1%d8%a7%d8%ad%db%8c-%d9%82%d8%a7%d9%84%d8%a8-%d9%88%d8%b1%d8%af%d9%be%d8%b1%d8%b3-%d8%aa%d9%85%d8%a7%d9%85%db%8c-%d9%82%d8%b3%d9%85%d8%aa-%d9%87%d8%a7-%d9%82/}{Link to the complete tutorials on WordPress theme design}
    \end{subsection}
    
\end{section}

\begin{section}{Education}

   \begin{subsectionnobullet}{Bachelor of Computer Engineering}{Software Engineering Concentration}{Islamic Azad University , Karaj}{2018 - 2023}
        \italicitem{Pursuing a Bachelor's degree in Computer Engineering with a specialization in Software Engineering.}
    \end{subsectionnobullet}
    
\end{section}

\begin{section}{Projects}
    \begin{subsection}{MrOlympiad}{Full-Stack Developer}{Apr 2019 - Nov 2020}{Contract}
     		\item Full-stack development using PHP, JavaScript, jQuery, and MySQL.
		\item	The MasterAlampiad website serves as a question bank and online exam platform for student Olympiads.
		\item	The implementation of this website began in early 2019 and was completed in two phases: question bank and online exams by the end of 2020.
		\item	Key features of the website include the sale of question packages, automatic exam generation using the question bank repository, personalized question categorization by students, and providing a panel for educational institutions to manage students and conduct online exams.
		\item	Upon the website's initial launch, approximately 1,000 registrations were recorded. Today, MasterAlampiad is widely recognized and popular among student Olympians

    \end{subsection}
    \begin{subsection}{Zama Chain Clothing Store Management Software}{Full-Stack Developer}{Apr 2022 - Aug 2022}{Contract}
     		\item Developed a web application for a chain of clothing stores with five active branches.
		\item The software included various features such as invoice generation, branch expense tracking, employee assistance recording, inventory management, stock rotation, sending discount codes to customers, customer loyalty program, profit and loss calculation, sending notifications for product depletion in stores or warehouses, employee salary calculation, and more.

    \end{subsection}

\begin{subsection}{Management Portal for Karaj Startup House}{Full-Stack Developer}{Jul 2022 - 2023 Feb}{Contract}
     		\item Developed a comprehensive management portal for Khaneh Startup Karaj, a co-working space located in Alborz province.
		\item Recognizing the challenges arising from the increasing population within the environment and the need for more efficient management of the pre-accelerator, the management portal was designed to cater to the specific needs of the members.
		\item The portal covers all processes, starting from the moment a team or individual enters the complex until their departure, providing a seamless experience throughout their stay.
		\item Implemented features that streamline various tasks such as registration, access management, resource allocation, event scheduling, and communication within the community.
		\item Improved the overall efficiency of operations by automating manual processes and providing a centralized platform for members to access relevant information and services.
		\item Collaborated closely with the Khaneh Startup Karaj team to gather requirements, design user-friendly interfaces, and ensure seamless integration with existing systems.
    \end{subsection}


    
\end{section}


\sectiontable{Technical skills}{
    \entry{Programming Languages}{PHP, Java}
    \entry{Web Development}{HTML, CSS, JavaScript, jQuery, Angular, Symfony}
    \entry{Databases}{MySQL, ORACLE}
\entry{Frameworks}{Spring Boot,  Symfony}
    \entry{Version Control}{Git}
	\entry{Project Management}{Agile methodologies (Scrum)}
%\entry{Soft}{Problem Solving and Analytical Thinking , Strong Communication and Collaboration Skills}
}


\sectiontable{Awards \& Coursework}{
 \entry{Digikala Educational Cup Competition }{Third Place \hfill \textit{2020}}
    \entry{PHP Certificate}{Quera College - Perfect Score - \href{https://quera.org/certificate/OYWeHltR/}{link} \hfill \textit{2020}}
\entry{Winter Camp, DigiNext}{Advanced tech and skills development. \hfill \textit{Winter 2022}}
}

\end{document}
